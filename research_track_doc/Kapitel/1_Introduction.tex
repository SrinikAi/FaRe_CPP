\chapter{Introduction}

Robots are increasingly vital in various industries, especially seen in factories where they assist with or completely take over tasks that are dangerous or tedious for humans. One such dangerous task is extinguishing fire when there are fire outbreaks  n large-scale industries and warehouses. These fire accidents are not only unsafe but also costly, especially when traditional firefighting methods are used. This is where autonomous firefighting robots come into play. These robots are specially designed to handle dangerous situations where it's highly unsafe for humans. Their ability to operate in hazardous conditions makes them valuable for industrial safety. 

Fire incidents are highly unpredictable, with their occurrence, location, and behavior changing in ways that are hard to anticipate. This unpredictability makes it challenging for standard methods to be effective, while even it's not easy for AI to make informed decisions in typical situations.Especially autonomous robots like Fire bots when deployed in real environments they must effectively keep surveillance on the site, and make informed decisions according to the current state of the environment using various sensors equipped to theses robot.

The critical nature of fire incidents demands not just any response, but a smart, strategic one. This is where the integration of a sophisticated task planner and surveillance algorithm becomes essential. 
The task planner and surveillance algorithm are crucial in empowering these robots to navigate complex environments and make real-time decisions. A well-designed task planner allows the robot to efficiently map its course of action, prioritizing tasks based on urgency and safety. It orchestrates the robot's movements and interventions, ensuring that every action taken is optimal under the given circumstances.

Together, the task planner and surveillance algorithm form the core of an autonomous firefighting robot's intelligence. They enable these robots not just to act, but to act wisely, making them indispensable in industrial fire safety strategies. Our research aims to enhance these systems further, pushing the boundaries of what autonomous firefighting robots can achieve, thereby significantly reducing the risk and impact of industrial fire incidents.









