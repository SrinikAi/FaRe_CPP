%=======================================================================

% Template for Thesis at Computer Sciencte at
% Ostfalia University of Applied Science
% Edit: Dirk J. Lehmann, Kai Michael Blum

%=======================================================================
\documentclass[
		11pt,
		a4paper, 
		oneside, % twoside use two sided for book layout
		% openany,
		toc=listofnumbered,
		toc=bibliography, 
		toftofoc,
		headings=small,
		headings=twolineappendix
]{scrbook}
\usepackage[utf8]{inputenc}

%=======================================================================

\newif\ifdraft
\draftfalse % Sagt aus, dass dieses Dokument ein Entwurf ist. Somit wird todonotes aktiviert. Zum deaktivieren diese Zeile auskommentieren oder auf \draftfalse setzen.

%=======================================================================

\input{Meta/praeampel}

%=======================================================================

\newcommand{\documenttitle}{Task Planning and Surveillance Algorithm For Autonomous FireBot  }
\newcommand{\documentname}{Srinivas Kachavarapu}
\newcommand{\documentmanr}{1002723}
\newcommand{\documentmodul}{}
\newcommand{\documentstudiengang}{Digital Technologies}
\newcommand{\documentachievement}{zur Erlangung des akademischen Grades:\\Master of Science\\}




\newcommand{\documentbetreuer}{Prof. Dr.-Ing. Reinhard Gerndt}
\newcommand{\documentbetreuerzwei}{Prof. Tobias Dörnbach, PhD}
\newcommand{\documentsemester}{2022}

\newcounter{documentfakultaet}
\setcounter{documentfakultaet}{1} % 0 = Bau Wasser Boden, 1 = Informatik

%=======================================================================


\begin{document}	

    % = = = = =	= = = = = = = = = = = = = = = = = = = = = = = =
	
	\input{Kapitel/0_0_Deckblatt}
    
    % = = = = =	= = = = = = = = = = = = = = = = = = = = = = = =	
	
	\frontmatter
	\pagenumbering{Roman}
	\ifnum\value{documentfakultaet}=1
		\input{Kapitel/0_erklaerung}
		\markboth{Erklärung}{}
	\fi
    
    % = = = = =	= = = = = = = = = = = = = = = = = = = = = = = =
	
	\chapter*{Abstract}

Dealing with fire incidents, especially in large recycling centers, poses significant challenges due to their unpredictability and potential for extensive damage. Traditional firefighting and detection methods often fall short, risking human safety and causing environmental pollution. To address this, we propose the use of AI-integrated Fire bots that can make intelligent decisions based on real-time environmental data through various sensors. These Fire bots should be able to: 1) monitor and patrol areas to  potential fire hazards, 2) detect and promptly extinguish any ongoing fires, and 3) urgently respond to external fire warnings by navigating to the location for inspection or fire suppression.

A crucial aspect of these Fire bots effectiveness lies in sophisticated high-level planning. In this work, we begin by presenting existing methodologies in robotic task planning and highlight a significant gap in the literature: the lack of specialized surveillance algorithms for patrolling in dynamic industrial environments. Addressing this gap, we introduce a novel hybrid task planning module. This module uniquely combines a task planner with an Surveillance algorithm, tailored for fire surveillance and response. To the best of our knowledge, this integration represents the first of its kind, empowering robots to make informed decisions in complex scenarios and implement an efficient surveillance system.The proposed algorithms are tested both in simulation using Gazebo and turtle bot.

In this study, we introduced the Hybrid Planner, an innovative approach to enhance autonomous fire-extinguishing robotics. Our findings reveal significant advancements in surveillance systems that can patrol  and mitigating fire hazards more effectively. Despite its potential, the Hybrid Planner faces challenges related to environmental adaptability and the cost of advanced technologies. Future work will focus on refining these areas, aiming to make this solution more robust and accessible. 
    
    % = = = = =	= = = = = = = = = = = = = = = = = = = = = = = =	
	
	% disable header
	\fancyhead[L]{\chaptermark}  % L -> Left part
	\fancyhead[RO]{\chaptermark} % RO -> Right part on Odd pages
	
	\tableofcontents % Inhaltsverzeichnis
	
	% = = = = =	= = = = = = = = = = = = = = = = = = = = = = = =
	
	\chapter{List of Abbreviations}
%
\begin{itemize}
     \item
        TPSA - Task Planner and Surveillance Algorithm
    
    \item
       HTN - Hierarchical Task Networks
    \item 
       RRT - Rapidly Exploring Random Trees
    \item
       TPSA - Task Planning and Surveillance Algorithm
    \item
       FBE - Frontier Based Exploration
    \item
       WFD - Wave-front Frontier Detector 
    \item
       LOS - Line Of Sight
    \item 
       WPO - Way-Point Optimization
    \item 
       GRASP - Greedy Randomized Adaptive Search Procedure 
    
    
\end{itemize}







	
	% = = = = =	= = = = = = = = = = = = = = = = = = = = = = = =
	
	\mainmatter
	
	% reanable header
	\fancyhf{}
	\rhead{\nouppercase\rightmark}
	\lhead{\nouppercase\leftmark}
	
	\fancypagestyle{plain}{}
	%\fancyhead[LE]{\chaptermark} % Chapter in header Left
	%\fancyhead[RE]{\chaptermark} % Page number in header Right
	%\fancyfoot[LE]{\thepage}     % LE -> Left part on Even pages
	\fancyfoot[RO]{\thepage}     % RO -> Right part on Odd pages
	
	% = = = = =	= = = = = = = = = = = = = = = = = = = = = = = =
	
	% reset list of used acronyms
	\acresetall
	
	% = = = = =	= = = = = = = = = = = = = = = = = = = = = = = =
	
	% Kapitel einfügen
	\chapter{Introduction}

Robots are increasingly vital in various industries, especially seen in factories where they assist with or completely take over tasks that are dangerous or tedious for humans. One such dangerous task is extinguishing fire when there are fire outbreaks  n large-scale industries and warehouses. These fire accidents are not only unsafe but also costly, especially when traditional firefighting methods are used. This is where autonomous firefighting robots come into play. These robots are specially designed to handle dangerous situations where it's highly unsafe for humans. Their ability to operate in hazardous conditions makes them valuable for industrial safety. 

Fire incidents are highly unpredictable, with their occurrence, location, and behavior changing in ways that are hard to anticipate. This unpredictability makes it challenging for standard methods to be effective, while even it's not easy for AI to make informed decisions in typical situations.Especially autonomous robots like Fire bots when deployed in real environments they must effectively keep surveillance on the site, and make informed decisions according to the current state of the environment using various sensors equipped to theses robot.

The critical nature of fire incidents demands not just any response, but a smart, strategic one. This is where the integration of a sophisticated task planner and surveillance algorithm becomes essential. 
The task planner and surveillance algorithm are crucial in empowering these robots to navigate complex environments and make real-time decisions. A well-designed task planner allows the robot to efficiently map its course of action, prioritizing tasks based on urgency and safety. It orchestrates the robot's movements and interventions, ensuring that every action taken is optimal under the given circumstances.

Together, the task planner and surveillance algorithm form the core of an autonomous firefighting robot's intelligence. They enable these robots not just to act, but to act wisely, making them indispensable in industrial fire safety strategies. Our research aims to enhance these systems further, pushing the boundaries of what autonomous firefighting robots can achieve, thereby significantly reducing the risk and impact of industrial fire incidents.










	% ----------------------------------------------------------
% ----------------------------------------------------------
\chapter{Literature Study}

We begin by presenting the task planning and surveillance algorithms and discuss about the issues, challenges, and current practices for each separately.


% ----------------------------------------------------------
\section{Task Planning}
\label{sec:Task Planning}
Task planning in robotics is very broad area where no one fits all solution exists,  many  aspects of domain, as well as  operational requirements, have effect on algorithms that are designed. In this research, we explore various common approaches to task planning as documented in literature, culminating in a novel Hybrid approach that addresses the specific challenges outlined in paper \cite{Mansouri2021Combining}.Among the many of strategies, such as Sequential Task Planning, Hierarchical Task Planning \cite{Kosak2020MapleSwarm}, Temporal Planning, Probabilistic Task Planning, and Reactive Task Planning, each has its strengths and limitations. However, in the context of a Fire bot scenario, task planning demands a more nuanced approach. Here, a dual-layered strategy becomes essential for autonomous decision-making: 1)operates at global level and decomposes the main tasks  to sub tasks to achieve overall complicated task, 2) micro level, making quick decisions in response to immediate changes in the environment or system state.

Therefore, a hybrid task planner, combining elements of Hierarchical and Reactive planning, is particularly well-suited for this domain. This hybrid system effectively merges the structured decomposition of tasks at a higher level with the adaptive, responsive execution of tasks based on real-time environmental conditions and system states. 







% ----------------------------------------------------------
\section{Surveillance Algorithm}
\label{sec:Surveillance Algorithm}

While numerous studies focus on robotics and autonomous navigation, particularly in mapping unknown environments \cite{s23104766}\cite{6847303}, there is a conspicuous absence of literature specifically targeting surveillance algorithms designed for scouting purpose in dynamic and hazardous environments such as those encountered by firefighting robots.
Existing literature predominantly addresses general mapping techniques like frontier-based exploration \cite{7276723} . However, these methodologies are not explicitly designed for the detailed and rapidly changing conditions of a patrolling task in fire surveillance scenario.The adaptability of frontier-based exploration methodologies, traditionally used for mapping unknown environments, is being reconsidered for surveillance in dynamic settings like firefighting. This shift in application addresses the lack of specialized surveillance strategies for such challenging conditions. By reworking the principles of established mapping techniques, a new, tailored algorithm is proposed to effectively meet the demands of firefighting scenarios, filling a notable gap in current research





% ----------------------------------------------------------

	\chapter{Task Planning}

\section{Introduction}
\label{sec:Introduction}


Task planning is a comprehensive strategy that instructs a robot on how to execute specific tasks to fulfill its primary goal. For robots such as Fire-Bots, which are designed for fire detection and suppression, task planning is crucial. It involves not only the identification and execution of tasks but also the optimal sequencing and timing of these actions. This is particularly vital in emergency situations where adaptability to dynamic environments is required.
In such complex and evolving scenarios, a hybrid task planner is typically employed. This approach integrates two principal components:

\textbf{1. Task Decomposition}: 
This involves breaking down complex tasks into simpler, manageable sub-tasks that the robot can execute efficiently. For Fire-Bots, this could involve segmenting a fire containment strategy into stages such as approach, assess, contain, and extinguish.

\textbf{2. Task Allocation}: 
This focuses on assigning these decomposed tasks to the robot or a fleet of robots, based on their capabilities, location, and the urgency of the tasks. In the context of our Fire-Bot, task allocation prioritizes fire detection and suppression while managing critical resources like battery life and water supply.

The hybrid task planner for the Fire-Bot, as discussed previously, operates within a system that merges hierarchical decision-making with reactive controls. It enables the robot to execute pre-planned actions for routine surveillance while remaining primed to react spontaneously to fire detection's or low battery alerts. This dual-layered strategy ensures that the Fire-Bot responds effectively to emergencies, maintaining a balance between its surveillance duties and its vital role in fire detection and extinguishing.



\section{Approach}
\label{sec:Approach}

In this part, we'll talk about how our Hybrid Planner works, which we introduced in section 3.1. Our method uses a mix of a step-by-step (Hierarchical) plan and an on-the-spot (Reactive) plan. This setup is designed to quickly adapt to changes while the robot is on fire watch duty, as shown in Figure 1

\label{sec:Approach}
\begin{figure}[h]
  \centering
  \includegraphics[width=0.9\textwidth, height=0.5\textheight]{Bilder/flow_chart.png}
  \caption{Flow chart for Hybrid Planner}
  \label{fig:flowchart}
\end{figure}

We break down the main goal into smaller tasks called Modes, such as Surveillance Mode for regular patrolling, Charging Mode for recharging the battery, and Extinguishing Mode for putting out fires. The planner can switch between these modes based on what's happening with the robot and any alerts it might get.

So, if the robot is going around checking for fires and suddenly finds one, it can switch to Extinguishing Mode to deal with the fire right away. Also, if the battery starts running low, the robot knows to switch to Charging Mode and navigate to base station to power up again. This way, Hybrid planner helps the robot do its job effectively, handling both its regular patrolling and any emergencies that come up.

\section{Implementation}



\begin{figure}[H]
\begin{algorithm}[H]
\DontPrintSemicolon
\SetKwFunction{FMoveToGoal}{MoveToGoal}
\SetKwFunction{FStopRobot}{StopRobot}
\SetKwFunction{FHandleFireDetection}{HandleFireDetection}
\SetKwFunction{FHandleLowBattery}{HandleLowBattery}
\SetKwFunction{FExecuteFireExtinguishing}{ExecuteFireExtinguishing}
\SetKwFunction{FExecuteCharging}{ExecuteCharging}
\SetKwFunction{FSwitchToSurveillanceMode}{SwitchToSurveillanceMode}
\SetKwFunction{FExplore}{Explore}
\SetKwProg{Fn}{Function}{:}{}
\Fn{\FMoveToGoal{x, y}}{
    \KwData{Goal coordinates \( x, y \)}
    \KwResult{Moves the robot to the specified coordinates}
    Send goal to action server with coordinates \( x, y \)\;
    Set robot state to "moving\_to\_goal"\;
}
\Fn{\FStopRobot{}}{
    \KwResult{Stops the robot by canceling its current goal}
    Publish cancel message to action server\;
    Set robot state to "idle"\;
}
\Fn{\FHandleFireDetection{}}{
    \KwResult{Handles the robot's behavior when fire is detected}
    \If{robot state is "idle" and battery level $\geq$ 10}{
        \FExecuteFireExtinguishing{}\;
    }
}
\Fn{\FHandleLowBattery{}}{
    \KwResult{Handles the robot's behavior when battery is low}
    \If{robot state is not "charging"}{
        \FExecuteCharging{}\;
    }
}
\Fn{\FExecuteFireExtinguishing{}}{
    \FStopRobot{}\;
    Move to the location of the fire\;
}
\Fn{\FExecuteCharging{}}{
    \FStopRobot{}\;
    Move to the charging station\;
}
\Fn{\FSwitchToSurveillanceMode{}}{
    \KwResult{Switches the robot to surveillance mode if it is idle}
    \If{robot state is "idle"}{
        \FExplore{}\;
    }
}
\Fn{\FExplore{}}{
    \KwResult{Runs the surveillance algorithm and optimizes the surveillance path}
    Run Exploration and set goals\;
    Run WayPointOptimizer and get best path\;
    \For{each goal in best path}{
        \If{fire detected or battery level $<$ 10}{
            break\;
        }
        \FMoveToGoal{goal.x, goal.y}\;
        Wait until goal is reached or failed\;
    }
    \If{not fire detected and battery level $\geq$ 10}{
        \FSwitchToSurveillanceMode{}\;
    }
}
\caption{Robot Controller Functions}
\end{algorithm}
\end{figure}

	\chapter{Surveillance Algorithm}


\section{Introduction}
\label{sec:Introduction}

In the realm of robotics, especially for those designed for emergency response and public safety, the implementation of a sophisticated surveillance algorithm is crucial. A fire bot, a type of robot specifically designed for firefighting and scouting in hazardous environments, is a prime example where such an algorithm is not just beneficial, but essential. Surveillance algorithm equips fire bots with the capability to dynamically scan and interpret their surroundings, facilitating real-time decision-making and strategy formulation.

As we talked about in section 2.2, there isn't much detailed research on using these surveillance systems in fire bots. But, there are some methods for exploring unknown places, like Frontier-Based Exploration [3] and RRT Exploration [4], that are similar to what we're trying to do. These methods are a good starting point for us. In this chapter, we're going to look at one of these exploration methods and think about how we can change it to make a surveillance system that fits what fire bots need.

\begin{figure}[h]
  \centering
  \includegraphics[width=0.9\textwidth, height=0.5\textheight]{Bilder/FBE1.png}
  \caption{Frontier Based Exploration [3]}
  \label{fig:FBE1}
\end{figure}

\begin{itemize}
    \item \textbf{Unknown Region} is the territory that has not been covered yet by the robot’s sensors.
    \item \textbf{Known Region} is the territory that has already been covered by the robot’s sensors.
    \item \textbf{Open-Space} is a known region which does not contain an obstacle.
    \item \textbf{Occupied-Space} is a known region which contains an obstacle.
    \item \textbf{Occupancy Grid} is a grid representation of the environment. Each cell holds a probability that represents if it is occupied.
    \item \textbf{Frontier} is the segment that separates known regions from unknown regions. Formally, a frontier is a set of unknown points that each have at least one open-space neighbor.
\end{itemize}
The terminology described above is shown in Figure.2, shows implementation of Frontier Based Exploration FBE[3] for mapping unknown environments. It utilizes  Wavefront Frontier Detector[7] Algorithm to find frontiers and then use gmapping, that implements Simultaneous Localization and
Mapping (SLAM) [5]and move base [6], a planning algorithm package to plan a path to move to the nearest frontier iteratively till the complete environment is mapped as shown in Figure.3.

\begin{figure}[h]
  \centering
  \includegraphics[width=0.9\textwidth, height=0.5\textheight]{Bilder/frontier_grids.png}
  \caption{Occupancy grids during exploration[3]}
  \label{fig:FBE2}
\end{figure}

\section{Approach}
\label{sec:Approach}

In this section, we build upon the insights gained from Section 4.1 about the Frontier-Based Exploration (FBE)  method.We will continue to discuss how we can develop a surveillance algorithm for robots.

In FBE, a key step is choosing the right frontiers. Frontiers are the locations on the occupancy grid that mark the boundary between areas the known region and unknown region. The Wavefront Frontier Detector [7] is used in FBE [3] to pick these frontiers. The robot then chooses the frontier closest to its starting position as its  target goal and uses a path planning algorithm to reach it. By doing this over and over, the robot fully explores the map, updating the occupancy map to show free space and space with obstacles in unknown region.

In our surveillance algorithm for robots, selecting the right frontiers is a key step. Unlike Frontier-Based Exploration (FBE), where any unexplored frontier is a target, our algorithm focuses on choosing frontiers that maximize surveillance coverage. We start with an occupancy map showing free, occupied, and unknown spaces.

\begin{figure}[h]
  \centering
  \includegraphics[width=0.7\textwidth, height=0.4\textheight]{Bilder/frontier.png}
  \caption{Frontier Boundary}
  \label{fig:frontier}
\end{figure}

The robot's surveillance range determines the potential frontiers, essentially the areas just beyond what the robot can currently see. We then select a target frontier from these options, prioritizing those that expand the surveillance area the most. This approach ensures that each movement of the robot is strategically made to enhance its surveillance effectiveness, as shown in Figure 4.

This method prioritizes strategic surveillance over mere exploration, choosing target frontiers that offer the greatest increase in monitored area within the robot's visual field.


Description of Figure.4:
\begin{itemize}
    \item \textbf{Explored Region} is the area  that has been covered  by the robot’s sensors (orange region).
    \item \textbf{Surveillance Range} is the range that senor can keep surveillance from robot's position.
    \item \textbf{Free-Space} is a known region which does not contain an obstacle.
    \item \textbf{Occupied-Space} is a known region which contains an obstacle.
    \item \textbf{Occupancy Grid} is a grid representation of the environment. Each cell holds a probability that represents if it is occupied.
    \item \textbf{Frontier} is the segment that separates explored regions from free space Formally.
\end{itemize}


\section{Implementation}
\label{sec:Implementation}

\subsection{Line Of Sight}
\label{sec:line of sight}

In the Frontier-Based Exploration (FBE) algorithm, a significant feature is the Wave Frontier Detection method, which identifies the boundaries between areas already explored and those that are not yet explored. This technique uses laser data to locate new frontiers, guiding the robot to unexplored regions for effective environmental mapping.

Our surveillance algorithm, while drawing inspiration from Wave Frontier Detection, diverges in its core objective to align with the specific demands of surveillance. In contrast to FBE's focus on uncovering unexplored areas, our approach concentrates on maximizing the efficiency of the surveillance area. This is achieved through a Line of Sight approach combined with using the surveillance range as a parameter to assess the area within the vicinity of the robot

\begin{figure}[h]
  \centering
  \includegraphics[width=0.7\textwidth, height=0.3\textheight]{Bilder/LOS.png}
  \caption{occupancy grid on left with out line of sight , and on right with line of sight}
  \label{fig:LOS}
\end{figure}

This adapted methodology functions similarly to a laser scan in a WFB. It evaluates the surveillance gain by considering the area within the robot's range, excluding regions obstructed by obstacles. Thus, unlike the FBE where the goal is to identify and move towards unexplored frontiers, our surveillance algorithm prioritizes efficient coverage of the observable area. It ensures that the chosen areas for surveillance are unblocked and within clear sight of the robot’s sensors, enhancing the quality and reliability of the surveillance data collected.





\begin{figure}[H]
\begin{algorithm}[H]
\DontPrintSemicolon
\SetKwFunction{FLineOfSight}{LineOfSight}
\SetKwProg{Fn}{Function}{:}{}
\Fn{\FLineOfSight{grid\_map, start, end, unoccupied\_value}}{
\KwData{Grid map \( grid\_map \), start point \( start \), end point \( end \), value for unoccupied cells \( unoccupied\_value \)}
\KwResult{frontier location if it's in line of sight}
\( x0, y0 \) \(\leftarrow\) start\;
\( x1, y1 \) \(\leftarrow\) end\;
\( dx \) \(\leftarrow\) \( |x1 - x0| \)\;
\( dy \) \(\leftarrow\) \( -|y1 - y0| \)\;
\( sx \) \(\leftarrow\) \textbf{if} \( x0 < x1 \) \textbf{then} \( 1 \) \textbf{else} \( -1 \)\;
\( sy \) \(\leftarrow\) \textbf{if} \( y0 < y1 \) \textbf{then} \( 1 \) \textbf{else} \( -1 \)\;
\( err \) \(\leftarrow\) \( dx + dy \)\;
\While{True}{
    \lIf{\(x0 = x1\) \textbf{and} \(y0 = y1\)}{\Return{True}}
    \lIf{\(grid\_map[x0, y0] \neq unoccupied\_value\)}{\Return{False}}
    \( e2 \) \(\leftarrow\) \( 2 \times err \)\;
    \If{\(e2 \geq dy\)}{
        \( err \) \(\leftarrow\) \( err + dy \)\;
        \( x0 \) \(\leftarrow\) \( x0 + sx \)\;
    }
    \If{\(e2 \leq dx\)}{
        \( err \) \(\leftarrow\) \( err + dx \)\;
        \( y0 \) \(\leftarrow\) \( y0 + sy \)\;
    }
}
\Return{\(x1,y1\)}\;
}
\caption{Function LineOfSight}
\label{alg:line_of_sight}
\end{algorithm}
\end{figure}


\subsection{Gain Function}
\label{sec:gain function}
Gain Function as a key component that evaluates and selects the most advantageous frontier based on the maximization of the surveillance area.

\begin{equation}
F = \{f_1, f_2, \ldots, f_n\} \quad \text{and} \quad G = \{g_1, g_2, \ldots, g_n\}
\end{equation}

Here, \( F = \{f_1, f_2, \ldots, f_n\} \) represents the set of potential frontiers, and \( G = \{g_1, g_2, \ldots, g_n\} \) represents the set of corresponding subgraphs for each frontier.
\begin{equation}
A_i = \left( \sum_{i,j} \left[ g_i[i, j] = \text{s} \right] \right) \times r^2
\end{equation}

Here, \(A_i\) represents the area of each sub graph from each frontier point,  S  represents state of explored  cells in graph, r is the resolution of cell.



\begin{equation}
G(f) = \max_{f \in F} A(f_i)\rightarrow f_i 
\end{equation}


The gain function \(G(f)\) , as articulated through these equations, stands at the core of this surveillance algorithm, intricately balancing the computational efficiency with the operational efficiency of the robot.


\subsection{Putting the Algorithm into Action}
\label{sec:Putting_the_Algorithm_into_Action}

Our surveillance algorithm is a type of Greedy Heuristic, which means it makes the best decision it can at each step, focusing on immediate benefits rather than long-term outcomes. Unlike algorithms that just find the shortest path, our approach aims to find a path that covers the largest surveillance area possible.

Here's how the algorithm works in action:

\textbf{1. Initial Setup and Iterations:}

The algorithm starts with the robot’s initial position as the first frontier. We define a surveillance area using the robot's current position as the center and its surveillance range to determine the boundary. This area is represented in Figure.4. We then use the Line of Sight approach to identify which parts of this area are visible and update our map accordingly, as shown in Figure.5.

\textbf{2. Finding and Evaluating Frontiers:}

In each subsequent iteration, we look for new frontiers just outside the current surveillance area and within the free space. The number of frontiers depends on several factors, including surveillance range, map size, and visibility. For each frontier \( f_i \) in the set \( F = \{f_1, f_2, \ldots, f_n\} \), we create a corresponding sub-graph \( g_i \) in the set \( G = \{g_1, g_2, \ldots, g_n\} \).


\begin{figure}[h]
  \centering
  \includegraphics[width=0.9\textwidth, height=0.5\textheight]{Bilder/frontiers.png}
  \caption{Randomly selected sub graphs in iteration - 2 shows  Surveillance area of 4 different frontiers with surveillance range of 4 meters.}
  \label{fig:Frontiers}
\end{figure}


\textbf{3. Applying the Gain Function:}

We apply the gain function to each sub-graph to find the one that offers the largest surveillance area. The selected sub-graph  frontier becomes our next way point, and sub graph is used for the following iteration.

\textbf{4. Repeating the Process:}

This process is repeated until we either reach a predefined number of iterations or cover all the free space, meaning there are no more frontiers left to explore. 

By focusing on maximizing the surveillance area at each step, the algorithm efficiently guides the robot through various way points, ensuring comprehensive coverage of the environment.

\section{Experimental result}

We further tested our algorithm in a real-world setting by creating a map of one of the university's blocks using SLAM (Simultaneous Localization and Mapping). This area, consisting predominantly of free space, spans approximately 110 square meters. We executed our surveillance algorithm on this map to demonstrate its practical effectiveness.


 \begin{figure}[h]
  \centering
  \includegraphics[width=0.7\textwidth, height=0.3\textheight]{Bilder/grid_maps.png}
  \caption{sub graphs at randomly selected iterations}
  \label{fig:LOS}
\end{figure} \

\(Figure 6\) illustrates how our surveillance algorithm covers the area, implementing the concepts discussed in \(Section 4.3\). Additionally, \(Table 1\) provides detailed insights into each iteration of the algorithm, including the area explored, the total number of frontiers identified, and the execution time for both single and multi-process setups. The results indicate that while the Gain Function is computationally intensive, running the algorithm in a multi-process environment is currently recommended for optimal performance.


\label{appendix:Heuristic Goals Estimation}

\begin{table}[ht]
\centering
\begin{tabular}{|c|c|c|c|c|c|}
\hline
Iteration & Goal      & Coverage Area   & Frontiers & CPU (s)  & multi process(s) \\ \hline
1         & (50, 50)  & 6.06            & 1         & 0.2277       
& 1                      \\ \hline
2         & (62, 67)  & 13.08           & 50        & 9.1336                 & 3                      \\ \hline
3         & (62, 88)  & 20.13           & 110       & 19.9720                & 5                      \\ \hline
4         & (68, 108) & 27.37           & 161       & 29.4274                & 8                      \\ \hline
5         & (89, 108) & 34.62           & 228       & 41.5267                & 11                     \\ \hline
6         & (105, 95) & 41.64           & 293       & 53.3847                & 14                     \\ \hline
7         & (82, 61)  & 48.37           & 352       & 64.1778                & 18                     \\ \hline
8         & (103, 61) & 54.09           & 392       & 71.0921                & 20                     \\ \hline
9         & (56, 125) & 59.08           & 414       & 74.8881                & 21                     \\ \hline
10        & (56, 146) & 64.06           & 448       & 80.8565                & 24                     \\ \hline
11        & (56, 167) & 69.61           & 461       & 83.2327                & 26                     \\ \hline
12        & (56, 188) & 74.53           & 488       & 88.0545                & 27                     \\ \hline
13        & (106, 120)& 79.03           & 502       & 90.4957                & 29                     \\ \hline
14        & (56, 209) & 83.05           & 495       & 89.0060                & 28                     \\ \hline
15        & (125, 89) & 86.16           & 501       & 90.0417                & 28                     \\ \hline
16        & (76, 170) & 88.19           & 533       & 95.6528                & 30                     \\ \hline
17        & (49, 104) & 89.01           & 526       & 94.4460                & 29                     \\ \hline
18        & (83, 80)  & 89.76           & 496       & 89.9992                & 28                     \\ \hline
19        & (84, 125) & 90.40           & 456       & 81.7659                & 26                     \\ \hline
20        & (85, 95)  & 90.99           & 431       & 77.2387                & 25                     \\ \hline
\end{tabular}
\caption{Table shows algorithm performance on each iteration for surveillance range of 2 meters on occupancy map in \(Figure.6\)}
\label{table:Table shows coverage area for each iterations, 
 selected frontiers, Exection time on cpu & multi processing}
\end{table} 



\section{Way-Point Optimization}

The approach we have discussed so far for determining way points in a surveillance area is indeed heuristic in nature. This means it aims to make the best possible decision at each step, focusing on immediate benefits rather than considering the entire sequence of actions in a long-term perspective. However, to clarify and refine our objective: our primary goal is to determine an optimal path that maximizes the surveillance area coverage.

In this context, we are not merely looking for a path that connects all designated points or goals. Instead, we are seeking a route that maximizes coverage of the surveillance area while also ensuring the overall path length is minimized. This dual objective requires a careful balancing act between covering as much area as possible and doing so in an efficient manner.

For implementing this optimization, we are employing the Greedy Randomized Adaptive Search Procedure (GRASP)[11]. GRASP is a well-known meta heuristic approach that is particularly effective for solving complex combinatorial problems. It operates in two main phases: a construction phase, where a feasible solution is built using a greedy randomized method, and a local search phase, where this solution is iteratively improved upon.

In the context of our problem, GRASP helps us in systematically exploring different configurations and sequences of way points, thereby enabling us to find a route that strikes the optimal balance between maximizing surveillance area coverage and minimizing travel distance. This approach is especially powerful in scenarios where the search space is large and the optimal solution is not immediately apparent, making it an ideal choice for our Way-Point Optimization task.

Thus, while our method is indeed heuristic, it is strategically designed to yield a solution that is as close as possible to the optimal, given the complexity and constraints of the problem at hand.




\begin{figure}[H]
\begin{algorithm}[H]
\DontPrintSemicolon
\SetKwFunction{FGRASP}{GRASP}
\SetKwFunction{FGreedyRandomizedConstruction}{GreedyRandomizedConstruction}
\SetKwFunction{FLocalSearch}{LocalSearch}
\SetKwFunction{FTwoOptSwap}{TwoOptSwap}
\SetKwFunction{FCalculateTotalDistance}{CalculateTotalDistance}
\SetKwProg{Fn}{Function}{:}{}
\Fn{\FGRASP{Goals, Alpha, MaxIterations}}{
\KwData{List of goals \( Goals \), randomness factor \( Alpha \), maximum number of iterations \( MaxIterations \)}
\KwResult{The best route found to cover all goals}

\For{k $\gets$ 1 \KwTo MaxIterations}{
    Solution $\gets$ \FGreedyRandomizedConstruction{Goals, Alpha}\;
    Solution $\gets$ \FLocalSearch{Solution}\;
    Cost $\gets$ \FCalculateTotalDistance{Solution}\;
    \If{Cost < BestCost}{
        BestSolution $\gets$ Solution\;
        BestCost $\gets$ Cost\;
    }
}
\Return{BestSolution}\;
}
\caption{Greedy Randomized Adaptive Search Procedure (GRASP)}
\end{algorithm}
\end{figure}

The GRASP (Greedy Randomized Adaptive Search Procedure) function orchestrates the overall process of finding an optimal path. It iteratively constructs a random solution using a greedy approach, then refines it through local search. The best solution across all iterations is returned as the final best path in \(Heuristic 3\)

\(Heuristic 4\) This is Greedy Randomized Construction function creates an initial solution by sequentially adding goals to the path. It evaluates and ranks possible next steps by their proximity, but introduces randomness by selecting from the top candidates, not just the nearest, which prevents the algorithm from being trapped in local optima.


\begin{figure}[H]
\begin{algorithm}[H]
\DontPrintSemicolon
\SetKwFunction{FGreedyRandomizedConstruction}{GreedyRandomizedConstruction}
\SetKwProg{Fn}{Function}{:}{}
\Fn{\FGreedyRandomizedConstruction{Goals, Alpha}}{
\KwData{List of goals \( Goals \), randomness factor \( Alpha \)}
\KwResult{A randomly constructed route}

CurrentPosition $\gets$ Goals[0]\;
Solution $\gets$ \{CurrentPosition\}\;
RemainingGoals $\gets$ Goals[1..end]\;
\While{RemainingGoals is not empty}{
    Distances $\gets$ list of distances from CurrentPosition to each point in RemainingGoals\;
    Sort Distances\;
    TopCandidates $\gets$ first Alpha percent of Distances\;
    Chosen $\gets$ randomly select a pair from TopCandidates\;
    Solution $\gets$ Solution + \{Chosen.Goal\}\;
    Remove Chosen.Goal from RemainingGoals\;
    CurrentPosition $\gets$ Chosen.Goal\;
}
\Return{Solution}\;
}
\caption{Greedy Randomized Construction Function}
\end{algorithm}
\end{figure}

The Local Search function takes an initial route and iteratively improves it. It uses the 2-opt swap strategy, which tries exchanging two legs of the route to see if a shorter overall distance can be achieved. This process repeats until no further improvements can be made  \(Heuristic 5\)




\begin{figure}[H]
\begin{algorithm}[H]
\DontPrintSemicolon
\SetKwFunction{FLocalSearch}{LocalSearch}
\SetKwFunction{FTwoOptSwap}{TwoOptSwap}
\SetKwFunction{FCalculateTotalDistance}{CalculateTotalDistance}
\SetKwProg{Fn}{Function}{:}{}
\Fn{\FLocalSearch{Solution}}{
\KwData{Initial route \( Solution \)}
\KwResult{An improved route after applying local search}

BestRoute $\gets$ Solution\;
BestCost $\gets$ \FCalculateTotalDistance{BestRoute}\;
Improved $\gets$ true\;
\While{Improved}{
    Improved $\gets$ false\;
    \For{i $\gets$ 1 \KwTo length(BestRoute)-2}{
        \For{k $\gets$ i+1 \KwTo length(BestRoute)}{
            NewRoute $\gets$ \FTwoOptSwap{BestRoute, i, k}\;
            NewCost $\gets$ \FCalculateTotalDistance{NewRoute}\;
            \If{NewCost < BestCost}{
                BestRoute $\gets$ NewRoute\;
                BestCost $\gets$ NewCost\;
                Improved $\gets$ true\;
            }
        }
    }
}
\Return{BestRoute}\;
}
\caption{Local Search Function}
\end{algorithm}
\end{figure}



\begin{figure}[h]
  \centering
  \includegraphics[width=0.7\textwidth, height=0.3\textheight]{Bilder/without_wpo.png}
  \caption{Path without way-point optimization (WPO) which maximizes surveillance area}
  \label{fig:waypoints without waypoint optimization (WPO) only maximizes surveillance area}
\end{figure}


\begin{figure}[h]
  \centering
  \includegraphics[width=0.7\textwidth, height=0.3\textheight]{Bilder/with_wpo.png}
  \caption{Path with way-point optimization (WPO) which maximizes surveillance area}
  \label{fig:waypoints with waypoint optimization (WPO) which also optimizes path that maximizes surveillance area}
\end{figure}












	\chapter{Integration}
The primary objective of this work is integrating the hybrid task planner with the surveillance algorithm is to create a robotic system capable of efficiently patrol and manage complex tasks in dynamic environments.This integration aims to leverage the strengths of both systems: the adaptability and responsiveness of the hybrid task planner and the thoroughness and coverage optimization of the surveillance algorithm.

The hybrid task planner brings a structured approach to managing tasks and making decisions based on a hierarchy of objectives. This ensures that the robot can handle both high-priority tasks (like emergency responses) and lower-priority tasks (like routine surveillance) without human intervention.

On the other hand, the surveillance algorithm optimizes the robot's path to maximize coverage of an area. When integrated with the task planner, this ensures that the robot's movements are not just reactive but are also aligned with the overarching goal of efficient area coverage.



\begin{figure}[h]
  \centering
  \includegraphics[width=0.9\textwidth, height=0.4\textheight]{Bilder/TPSA.png}
  \caption{TPSA Architecture}
  \label{fig:frontier}
\end{figure}

Integrated Task planner and Surveillance algorithm (TPSA) architecture as shown in \(Figure 7\) has the robot controller component that  encapsulates the task planning and decision-making logic. It monitors the robot's state, battery levels, and external triggers such as fire detection. Based on these inputs, the controller dynamically switches between different operational modes, including direct intervention (like extinguishing fires), recharging, and surveillance.

	\input{Kapitel/8_Future_work}

    % = = = = =	= = = = = = = = = = = = = = = = = = = = = = = =

    % Bilbliography
	\bibliographystyle{dinat} 
	\bibliography{literatur}
	
	% = = = = =	= = = = = = = = = = = = = = = = = = = = = = = =
	
	\ifnum\value{documentfakultaet}=0
		\input{erklaerung}
		\markboth{Erklärung}{}
	\fi
	
	% = = = = =	= = = = = = = = = = = = = = = = = = = = = = = =
	
    %\addtocontents{toc}{\cftpagenumbersoff{chapter}} % Keine Seitenzahl im Inhaltsverzeichnis
	\begin{appendices}              % Anhang Titel-Seite
	\end{appendices}
	
	\fancyfoot[R] {\thepage}        % Seitenzahl Right
	
	\appendix
	\input{Kapitel/100_Anhang}
	%\postappendix
	
	% = = = = =	= = = = = = = = = = = = = = = = = = = = = = = =
	
\end{document}