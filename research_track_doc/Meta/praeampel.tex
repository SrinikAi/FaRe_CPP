% In diesem Dokument werden die globalen \usepackage{}-Befehle eingefügt.
% Definition des Seitestils (Ränder, Schriftart, Abstände, usw.)

\usepackage[
		top=2.5cm, 
		bottom=2cm, 
		left=3cm, 
		right=2cm,
		includeheadfoot
]{geometry}

\usepackage[english]{babel}
\selectlanguage{english}
\usepackage[scaled]{Meta/uarial}     % Schriftart Arial
\usepackage[T1]{fontenc}        % bessere und richtige Schriftausgabe

% here for tables
\usepackage{here}

% appendix
\usepackage[toc,page]{appendix}
\renewcommand\appendixpagename{Appendix}
\renewcommand\appendixtocname{Appendix}

% chapter 14pt, section 12pt, subsection 11pt
\setkomafont{chapter}{\normalfont\bfseries\fontsize{14}{16.8}\selectfont}   
\setkomafont{section}{\normalfont\bfseries\fontsize{12}{14.4}\selectfont}   
\setkomafont{subsection}{\normalfont\bfseries\selectfont}     

\usepackage{etoolbox}
\makeatletter
% No breake in chapter
\patchcmd{\@@makechapterhead}{\endgraf\nobreak\vskip.5\baselineskip}{}{}{}


\let\my@chapter\@chapter
\renewcommand*{\@chapter}{%
	\addtocontents{lof}{\protect\addvspace{0pt}}%
	\my@chapter}

\makeatother

%\setuptoc{toc}		% Inhaltsverzeichnis im Inhaltsverzeichnis

\usepackage[titles]{tocloft} % Anhang im Inhaltsverzeichnis

% Abstand Chapter
\renewcommand*{\chapterheadstartvskip}{\vspace*{-\baselineskip}}
\renewcommand*{\chapterheadendvskip}{\vspace*{10pt}}

% durchgehende nummerierung bei Tabellen und Bildern
\usepackage{chngcntr}
\counterwithout{figure}{chapter}
\counterwithout{table}{chapter}

% Prefixe von Abbildungen und Tabellen
\addto\captionsngerman{\renewcommand{\figurename}{Fig.}}
\addto\captionsngerman{\renewcommand{\tablename}{Tab.}}
\renewcommand{\cftfigpresnum}{\figurename\ }
\renewcommand{\cfttabpresnum}{\tablename\ }

% Abstand zwischen Nummer und Name von Abbildungen und Tabellen in Verzeichnissen
\newlength{\mylenf}
\settowidth{\mylenf}{\cftfigpresnum}
\setlength{\cftfignumwidth}{\dimexpr\mylenf+1.5em}
\setlength{\cfttabnumwidth}{\dimexpr\mylenf+1.5em}

% Einrückung in Tabellen- und Abbildungsverzeichnis aufheben
\renewcommand{\cfttabindent}{0em}
\renewcommand{\cftfigindent}{0em}


\usepackage[bottom]{footmisc} % Fußnoten nach unten stellen

\usepackage{amsmath} % amsmath package verwenden
\usepackage{enumitem} % enumaration für listen modifizierbar machen
\usepackage{amsfonts} % amsfonts package verwenden
\usepackage[linesnumbered,ruled]{algorithm2e}
\SetAlgorithmName{Heuristic}{heuristic}{List of Heuristics}

\usepackage[noend]{algpseudocode}
\usepackage{pgfplots}
\pgfplotsset{compat=1.18}
\usepackage{listings}

\setlength{\parindent}{0em} % Absatzeinrückung
\setlength{\parskip}{0.5em}  % Absatzeinrückung

%1,5 facher Zeilenabstand
\usepackage{setspace}
\onehalfspacing			

% Grafiken einbinden
\usepackage{graphicx}

% mainmatter erweitern
% Kopfzeile auf auf seiten mit Kapitelstart
\newcommand{\oldmainmatter}{}
\let\oldmainmatter\mainmatter
\renewcommand{\mainmatter}{\oldmainmatter \renewcommand{\chapterpagestyle}{fancy}}

% Kopf und Fußzeilen
\usepackage{fancyhdr}
\pagestyle{fancy}
\setlength{\headheight}{14.5pt}
\setlength{\headsep}{0.7cm}
\fancyhf{}
\rhead{\nouppercase\rightmark}
\lhead{\nouppercase\leftmark}
\rfoot{Seite \thepage}

% Kapitel

\usepackage{csquotes}
\usepackage[square,numbers]{natbib} % Natbib für DINAT

% PDF Einstellungen
\usepackage[pdftex, pdfborder={0 0 0}]{hyperref}
%\hypersetup{
%	pdfauthor={\documentname\ \documentmanr},
%	pdftitle={\documenttitle}
%	pdfkeywords={Betreuer: \documentbetreuer}
%	}

% Besseres Aussehen in PDF
\usepackage{lmodern}

% Anhang zusätzliche einstellungen:
% Nummerierung ändern:
\usepackage{ifthen}

\newcommand{\oldappendix}{}
\let\oldappendix\appendix
\renewcommand{\appendix}{
	
	\oldappendix
	\newcommand{\oldchapter}{}
	\let\oldchapter\chapter
	\renewcommand{\chapter}[2][]{
		\ifthenelse{\equal{##1}{}}  {
			\oldchapter{##2}
		}{
			\oldchapter[##1]{##2}
		}    		% if title given
		\setcounter{page}{1}
	}    
	    
	\renewcommand{\thepage}{\Alph{chapter} \arabic{page}}
	
	%\cftaddtitleline{toc}{chapter}{\appendixname}{}
	\addtocontents{toc}{\protect\setcounter{tocdepth}{-1}}
	\hypersetup{bookmarksdepth=-1}
	\renewcommand*{\autodot}{:}% Doppelpunkt nur im Anhang
}

\newcommand{\postappendix}{
	\addtocontents{toc}{\protect\setcounter{tocdepth}{3}}
	\hypersetup{bookmarksdepth=3}
}


% Todo package ein- und ausschalten
\ifdraft
	\usepackage{todonotes}
\else
	\usepackage[disable]{todonotes}
\fi

% Abkürzungsverzeichnis
\usepackage[printonlyused]{acronym}

% Ränder testen
% \usepackage{showframe} % Auskommentieren, um die Layoutgrenzen anzuzeigen

% Testtexte
\usepackage{lipsum}
